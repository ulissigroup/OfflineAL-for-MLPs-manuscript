\documentclass[%
 reprint,
 amsmath,amssymb,
 aps,
]{revtex4-2}


\usepackage{graphicx}% Include figure files
\usepackage{dcolumn}% Align table columns on decimal point
\usepackage{bm}% bold math
\usepackage{glossaries}     % Enable the \newacronym and \gls commands to reference terms
\usepackage{commath}        % We have this here to use the \abs{} equation function
\usepackage{amssymb}
\usepackage{svg}
\usepackage{mathtools}
\usepackage{booktabs}
\usepackage{array}
\usepackage{makecell}
\usepackage{todonotes}
% \renewcommand\theadalign{bc}
\renewcommand\theadfont{\bfseries}
\usepackage{multirow}
\newcommand{\ang}{\mbox{\normalfont\AA}}
\newacronym{DFT}{DFT}{Density Functional Theory}
\newacronym{onlineal}{\onlineal}{Online Active Learning}
\newacronym{offlineal}{offlineal}{Offline Active Learning}

\newacronym{mlp}{MLP}{Machine Learning Potential}
\newacronym{QE}{QE}{Quantum Espresso}
\newacronym{VASP}{VASP}{Vienna Ab initio Simulation Package}

\begin{document}

\title{Supporting Information}

\author{Muhammed Shuaibi, Saurabh Sivakumar, Rui Qi Chen}
%  \email{mshuaibi@andrew.cmu.edu}
\author{Zachary W. Ulissi}%
 \email{zulissi@andrew.cmu.edu}
\affiliation{%
Department of Chemical Engineering, Carnegie Mellon University, Pittsburgh, PA 15213, United States
}%

\date{\today}
            
             
\section{Supplementary Information}
\renewcommand{\theequation}{S.\arabic{equation}}
\renewcommand\thefigure{S.\arabic{figure}}
\setcounter{figure}{0}  
\setcounter{equation}{0}  
\subsection{Interactive examples}
Several interactive Google Colab notebooks have been prepared to allow readers to conveniently explore the proposed methods. Accelerated structural relaxations and transition state calculations can be found at: \url{https://github.com/ulissigroup/Enabling-Robust-Offline-Active-Learning-for-MLPs}. Random query strategies are used to demonstrate the effectiveness of even the simplest of strategies. We encourage users to explore query and termination strategies that best suites their application of choice. DFT calculations are performed directly in the notebook examples via a GPU-enabled Quantum Espresso package.

\subsection{Morse parameters fitting}
AMP\textit{torch}, the code for this work, is available at Ref \cite{amptorch}. Additionally we provide interactive Google Colab \cite{examples} notebooks to allow users to easily explore alternative systems without the hassle of installation and dependency management. 

The Morse potential was selected for our primarily bonded, catalytic systems. Parameters of the Morse potential, $D_e$, $r_e$, and $a$, corresponding to well depth, equilibrium distance, and well width were computed in the following manner for a given element, X:
\begin{enumerate}
    \item Lone atomic energies, $E_X$, obtained through singlepoint DFT calculations;
    \item Diatomic atoms relaxed to obtain a relaxed state energy, $E_{X_2}$, and equilbrium distance, $r_e$.
    \item Well depth, $D_e$, is calculated as follows:
    \begin{equation}
        D_e = -(E_{X_2} - 2 * E_X)
    \end{equation}
    \item Diatomic bond stretched and corresponding DFT points fit to morse potential functional form (\ref{morse}) to obtain $a$ Figure(\ref{fig:morsefit}).
    \begin{equation}\label{morse}
    E_{morse} = D_e(e^{-2a(r-r_e)} - 2e^{-a(r-r_e)})
    \end{equation}
\end{enumerate}
 
To make use of the Morse potential for multi-element systems, linear mixing rules are utilized to compute element pair parameters. Adapted from Yang, et al.\cite{Yang2018} the Morse potential is rewritten and parameter combinations applied accordingly (
\begin{equation}\label{newmorse}
    E_{morse} = D_e(\exp[{-\frac{2C}{\sigma}(r-r_e)}] - 2\exp[{-\frac{C}{\sigma}(r-r_e)})]
\end{equation}

\begin{flalign}\label{fig:mixing}
    D_{AB} &= \sqrt{D_AD_B}\\
    r_{e, AB} &= \frac{r_{e, A} + r_{e, B}}{2}\\
    \sigma_{AB} &= \frac{\sigma_{A} + \sigma_{B}}{2}
\end{flalign}

Where $C = \ln{2}/(r_e - \sigma)$ and $\sigma$ corresponds to $E_{morse}(\sigma) = 0$. Although more sophisticate combination rules exist \cite{Yang2018}, the accuracy of our Morse potential is not crucial for the success of our framework as it is meant to provide some guidance to the model.

\subsection{Convergence}

The convergence of the \gls{offlineal} loop can be accelerated through the use of a learning rate scheduler. Figure \ref{fig:scheduler} compares the learning curves of AL frameworks with and without a learning rate scheduler, ceteris paribus. We demonstrate that a cosine annealing scheduler with warm restarts \cite{loshchilov2016sgdr} was able to assist the convergence by smoothing out the learning curve and requiring fewer training images to reach a similar level or error.

\begin{figure*}[!th]
    \centering
    \includegraphics[width=\textwidth]{figures/figure_S1.pdf}
    \caption{Morse parameters are obtained by fitting DFT points near the equilibrium distance to equation \ref{morse}. Sample fittings are illustrated for copper, carbon, and oxygen.} 
    \label{fig:morsefit}
\end{figure*}  

\begin{figure*}
    \centering
    \includegraphics[width=\textwidth]{figures/figure_S2.pdf}
    \caption{Offline-AL convergence of our BPNN $\Delta$-ML is compared with and without a learning rate scheduler. The use of a scheduler, particularly with small data, enables our framework to converge more reliably to the local minima.}
    \label{fig:scheduler}
\end{figure*}
\bibliography{paper}% Produces the bibliography via BibTeX.
\end{document}

